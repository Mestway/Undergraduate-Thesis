%!TEX root=..//thesis.tex

\chapter{实验评估}

我们将我们的转换工具实现为eclipse的一个插件,用来进行插件中的Project的转换操作。在实现中我们采用了以下部件:分析工具soot,AST操纵工具eclipse JDT。我们的项目页面在 \url{https://github.com/Mestway/Patl4J}。

在实验中,我们需要进行对于如下问题的评估:
\begin{enumerate}
\item 我们进行一个项目转换需要多少的规则来描述?
\item 在转换过程中我们需要解决多少的多对多问题?
\item 我们在转换过程中会进行多少错误提示?
\end{enumerate}
接下来的部分我们就通过数据来回答以上几个评估问题。

在评估过程中,我们采用了Jdom到Dom4j以及Google calendar v2到Google calendar v3之间的转换。

在这两个项目中,我们进行了Jdom中读写API之间的转换,以及gCalender中全部API的转换。我们书写一下规则使用的代码行数如下图所示:

\begin{center}
    \begin{tabular}{|c|c|c|c|c|}
        \hline
        Transformation & Rules & Classes & Methods & M-to-m\\
        \hline \hline
        Jdom/Dom4j & 84 & 12 & 77  & 12(14.3\%) \\
        \hline
        Calendar & 42 & 14 & 45 & 21(50.0\%) \\
      \hline
      \hline
      Total & 126 & 26 & 122 & 33(26.2\%) \\ 
        \hline
    \end{tabular}
\end{center}


由此可见代码转换行数是可以达到人为编程水平的,以及我们在不同的转换中M-to-m(多对多)转换是具有一定数量需要解决的。

使用了上述的这些转换规则,我们在几个客户端程序之中进行了API之间转换,列表如下:

\begin{center}
    \begin{tabular}{|c|c|c|c|c|c|}
        \hline
        Client & KLOC & Classes & Methods  & Case\\
        \hline \hline
        husacct & 195.6 & 1187  & 5977 & Jdom/Dom4j \\
        \hline
        serenoa & 12.2  & 52 & 523 & Jdom/Dom4j \\
        \hline
        openfuxml & 112.5 & 727 & 4098 & Jdom/Dom4j \\
        \hline
        clinicaweb & 3.9 & 74  & 213 & Calendar \\
        \hline
        blasd & 9.7 & 199 & 729 & Calendar  \\
        \hline
        goofs & 8.6 & 78 & 643 & Calendar  \\
      \hline
      \hline
      Total & 342.5 & 2317 & 12183 & --\\
      \hline
    \end{tabular}
\end{center}

这几个规则中,husacct是一个软件体系结构表现的检测工具,serenoa是一个用户交互界面的修改框架,openfuxml是一个XML的发布框架,他们都采用了Jdom作为内部的XML处理支持,我们在测试中将他们转换为使用Dom4j接口的XML处理平台。另外的cliniaweb是一个在线临床的诊断平台,blased是一个邮件系统,goofs是一个文件系统,他们都是用了google calendar v2作为其中的日历处理接口,我们在转换中将这些接口升级为v3版本。

这些项目的转换结果如下:

\begin{center}
    \begin{tabular}{|c|p{0.1\columnwidth}|p{0.1\columnwidth}|c|c|c|c|}
        \hline
        Client & CF & CL & U & W & MM \\
        % methods includes method definition, method invocation, and constructor
        % declarations includes method declarations and variable declaration
        \hline \hline
        husacct     & 42 & 852 & 0& 0(0\%) & 0(0\%)\\
        \hline
        serenoa     & 8 & 273 & 0& 3(1\%) & 9(3.3\%) \\
        \hline
        openfuxml   & 72 & 983 & 15 &  54 (6.6\%) & 2(0.2\%) \\
        \hline
        clinicaweb  & 5 & 81 & 8 & 0 (0\%) & 34(42\%) \\
        \hline
        blasd      & 5 & 26 &0& 7 ( 21.2\%) & 13(50\%) \\
        \hline
        goofs      & 13 & 100 & 27 & 13 (15.3\%) & 27(27\%) \\
      \hline
      \hline
      Total & 145 & 2315 & 50 & 77(3.2\%) & 85(3.7\%) \\ 
      \hline
    \end{tabular}
    \parbox{\columnwidth}{\it 
    CF =修改的文件数, CL= 修改的代码行数, U =
    无法进行转换的代码行数, W = 报转换错误的数目,
    其中的比例表示 W / (W + CL), 
    MM = 涉及M-M转换的代码行数,
  变分比表示 = MM / CL}
\end{center}

由此可见我们在转换过程中能够较好的处理M-M映射问题,同时能够比较完整的进行API转换工作。

关于项目实现以及使用,更多的消息以及更新请参见项目页面。